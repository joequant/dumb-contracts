\documentclass[10pt]{article}

\usepackage{geometry}
\geometry{letterpaper,tmargin=1in,bmargin=1in,lmargin=1.4in,rmargin=1.4in}

\begin{document}
\begin{center}
{\Large Bitcoin purchase process}
\end{center}

Bitcoin purchase process in Hong Kong.  This document is an evolving
document that describes the high volume bitcoin purchase process in
Hong Kong.  The process of buying and selling bitcoin is made
difficult by the lack of standard processes and agreements, and this
document is designed as a working description of standards and
processes.

\section{Pre-settlement process}

\begin{itemize}
\item Buyer issues non-binding letter of intent and proof of funds to
sellers broker.

\begin{itemize}
\item The letter of intent is a simple document describing the intent
  to buy bitcoin with a brief description of acceptable settlement
  methods and describe amounts
  
\item The proof of funds consists of a bank statement or a computer
screenshot with the identifying information removed (i.e. bank account
and account holder information removed)

The reason for doing a proof of funds early is that this turns out
that the buyer agent usually needs time to do the internal processes
necessary to get proof of funds.  
\end{itemize}

If necessary, buyer and seller issue pre-sales agreement allowing of
exchange of proof of coin and proof of funds.

\item Buyer and seller agree on price and settlement details

Our experience has been that the price can usually be rather quickly
agreed upon.  The difficulty has been lack of standard settlement
methods.

\item Seller sends buyer proof of coin doing satoshi test

\item Seller's agent can send interim memorandum of understandings`

\item Following documents are executed:

\begin{itemize}
\item Sales and purchase agreement between buyer and seller
\item Non-circumvention agreement to protect middlement
\item Fee protection agreement for middlemen
\end{itemize}

\item Buyer/seller performs due dillegence.  This may involve
  contacting escrow agents and other intermediaries to verify
  identity.
\end{itemize}
  

\section{Two mechanisms of transfers within Hong Kong}

\subsection{Back to back escrow}

\begin{itemize}
\item Buyer transfers funds to buyer escrow trust account.
\item Buyer escrow informs seller lawyer that funds have been transferred.
\item Seller escrow agent transfers bitcoin to buyers wallet.
\item Buyers escrow transfers funds to sellers escrow agent
\item It may be the case in which there are identical SPA's which are
  chained through middlemen
\end{itemize}

\subsection{Bank F2F}

\begin{itemize}
\item Buyer and seller agree to meet at bank VIP room
\item Buyer has bank issues series of cashier’s checks
\item Seller has bank authenticate cashier’s checks
\item Buyer puts cashier check on table
\item Seller transfers bitcoins
\item Seller takes cashiers check
\item Process repeats
\end{itemize}

Also we are working on funds transfers via bank letter of guarantee
and through Hong Kong trust companies.

\section{Difficulties and issues}

\subsection{POF/POC issues}

One biggest problems involve proof of funds and proof of coin.
Neither side is willing to provide information to the other side
without confirming from the first.

We've found in practice that it is not useful to start discussions
over settlement and discount before POF/POC issues can be resolved.
The reason for this is that because coins move quickly, if people
reach an agreement on other issues, but it stalls because of POF/POC
issues then the deal will fail.

\end{document}
